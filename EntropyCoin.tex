\documentclass[twocolumn]{article}
\begin{document}
\frenchspacing

\title{EntropyCoin: A Difficult-To-Manipulate Source of Entropy}

\author{
{\rm David Vorick}\\
Nebulous Labs
}
\author{
{\rm Luke Champine}\\
Nebulous Labs
}

\maketitle

\subsection*{Abstract}
EntropyCoin is a project designed to target an explicit problem: the deterministic and secure generation of entropy.
Currently, it is difficult for multiple hosts to generate 
The best known solutions require a large portion of participants to be honest and can only provide bounded random numbers.
EntropyCoin presents a lightweight solution for generating difficult-to-manipulate random numbers.

\section{Introduction}
Decentralized networks need to be able to cope with malicious behavior by a set of nodes.
Decentralized networks can require the generation of random numbers.
Consider the problem of the Random Byzantine Generals Election.
There are a set of Byzantine Generals trying to randomly elect a commander.
The election results should insure that
\begin{enumerate}
	\item The commander is elected at random.
	\item The random number cannot be easily manipulated by traitors.
	\item Manipulating the random number is expensive and uncertain.
\end{enumerate}

Existing solutions to this problem require both a high percentage of loyal generals and produce results that can be manipulated within some epsilon.
These solutions typically have each general generating a random outcome and then somehow combining the outcomes.
A different approach is needed.
% If possible, prove that internal random number generation will not work
% Explore the idea that traitors can withdraw from the process after waiting for a reveal.

The Bitcoin proof-of-work model provides this different approach.

% pick up here

EntropyCoin is intended to be used for deterministically generating random numbers as a group.
The numbers generated by EntropyCoin are public, and are not inteded to be used as entropy for secrets such as the seed to a private key.
Instead, the random numbers are to be used in public settings where randomness is required but secrecy is not.
An example would be randomly determining the address of a node in a network.
If an attacker can choose where a node is placed, certain attack vectors open, so secure public random number generation is required.
Maidsafe is one example of a network with such a need, Sia is a second example.

Bitcoin already provides such a source of entropy.
At present prices, the reward for producing a block on the bitcoin network is \$11,250.
The hash of a block header can serve as entropy for a network.
The most that such entropy could be manipulated is by producing multiple blocks and choosing the preferred random number.
Choosing a preferred random number means orphaning a valid Bitcoin block and sacrificing the 25 bitcoin reward.
Not only is it costly to sacrifice a reward, it's also unlikely that an attacker will be able to get two blocks in a row without a massive amount of mining infrastructure.
Entropy generated from block headers can therefore be assumed to be highly secure.

A network looking for entropy could use Bitcoin, however the Bitcoin blockchain is large and busy, having many transactions.
Furthermore, there is no internal value to Bitcoin, and no guarantee of the Bitcoin price.
Should the value of 25 bitcoins suddenly plummet, the security of the entropy is thrown into question.

EntropyCoin has been created from the same basic principles with these concerns and optimizations in mind.

\section{EntropyCoin as a Lightweight Blockchain}
The first optimization is to keep the EntropyCoin blockchain lightweight.
Each block, 1 EntropyCoin is mined.
Each transaction, an integral volume of entropycoins must be transferred.
This will keep dust off of the network.

Once generated, hashes do not need to be kept.
Therefore, old blocks are discarded and only the network state is kept.
The network state maintains a list of entropycoin owners and the volume that they own, nothing else.
Because transactions are integral, and only 1 entropycoin is created per block, the state will have at most one entropycoin owner per block.

The blockrate for EntropyCoin is set to 2.5 minutes, modeled after Litecoin.
This is to provide a constant stream of entropy while leaving plenty of time for network propagation.
Other models, such as a self-adjusting blockrate, have been rejected because they could open up attack vectors.
To maximize security, Entropycoin is kept rigid.

At one block per 2.5 minutes, the total number of blocks produced each year is 210,380.
At 200 bytes per wallet (a generous number, likely to be less than 64 bytes), that is a maximum network size of 42.1 mb per year.
Given that the majority of wallets are likely to have many coins, the actual size of the network should be below 10mb.

\section{Economic Incentive}
EntropyCoin does not try to compete as a way to trade value, so the value of entropy coin must be derived from elsewhere.
Instead, EntropyCoin derives value by being destructible.
Any wallet can release a transaction that destroys an integral volume of EntropyCoins.
It is assumed that various entities will have a vested interest in keeping entropy expensive to manipulate.
These entities are expected to contribute to the network by buying entropycoins and destroying them.
The other way to contribute is to mine entropycoins.

To maximize the incentive for mining on EntropyCoin, EntropyCoin will be merge mined with Bitcoin.
This means that Bitcoin miners can reap all of the standard profits of Bitcoin mining, as well as the profits from Entropycoin mining.

\section{Conclusion}
This relatively simple coin has been created as a lightweight solution to the problem of deterministic public entropy generation.

\end{document}
